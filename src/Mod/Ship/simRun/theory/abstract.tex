\chapter{Introduction}
\label{s:introduction}
%
\section{Objective}
%
The objective of this document is introduce briefly how the waves of the
seakeeping simulator are propagated.\rc
%
In the seakeeping simulator Boundary Elements Method (BEM) will be used,
that is detailed described in several books, like \citet{bem_2007}.
\citet{vinayan2007} gives a detailed description of the propagation
of waves in a 2D case, and is a good starting point.\rc
%
We will start briefly describing the governing equations in order to can
start working with the 2D problem. First the incident waves over our
computational domain will be described, introducing also the potential,
discussing then the BEM applied to this case. As we will see the Laplace
problem in the 2D case will not be really useful for us.\rc
%
After that we can start working in the 3D case, that is our real objective.
The incident waves will be rewritten, and the Laplace problem and the BEM
application purposed again.
%
\chapter{Governing equations}
\label{s:governing_equations}
%
Assuming no viscous fluid (that allows to transform Navier-Stokes
equations into Euler ones), and imposing an initial condition such
that\footnote{With no viscous fluid this condition is preserved along the
time}:
%
\begin{eqnarray}
	\rotational \bs{u} = 0
\end{eqnarray}
%
The fluid velocity derives from a scalar function potential $\phi$
%
\begin{eqnarray}
	\label{eq:governing_equations:v_potential}
	\gradient \phi = \bs{u}
\end{eqnarray}
%
Then the Navier-Stokes equations can be rewriten as a Laplacian problem
and Bernoulli equation:
%
\begin{eqnarray}
	\label{eq:governing_equations:laplace}
	\laplacian \phi = & 0
	\\
	\label{eq:governing_equations:bernoulli}
	p = & - \rho \left( \vert \bs{u} \vert^2 + g \bs{z} \right)
\end{eqnarray}
%
And in order to solve the Laplace problem \ref{eq:governing_equations:laplace}
we will use the BEM as described by \citet{bem_2007}.
%